\documentclass[10pt]{article}
\usepackage[usenames]{color} %used for font color
\usepackage{amssymb} %maths
\usepackage{amsmath} %maths
\usepackage[utf8]{inputenc} %useful to type directly diacritic characters
\begin{document}

Yaniv is a popular card game in Israel also known in Nepal by the name XX?

Here we describe the basics of the game and describe the statistics for successful decleration of a round

Definitions:   \\
$C_{i,j}$ - cards possessed by player $i$ as known to player $j$.  
$H_i$ - sum of values cards in hand of player $i$.    \\
$n_{\mathrm{cards}_i}$ - number of cards in the hand of player $i$.  \\


Our main objective is to calculate the probability that player $i$ possesses the lowest hand: $P_{i}(H_i = \mathrm{min}\{H\}) $.  \\

For simplicity we will commence with two players reducing our task to calculate  $P(H_j>H_i)$.  

\section{Two Players}

Assuming a player $i$ declared victory (known as calling Yaniv), we would like to know the probability of success. 
We assume $N$ cards are not known to the observer (player or outsider) and that player $j$ has $n_j$ cards in their hand. The number of possible hand card combinations are 
$\binom{N}{n_j}$. In the extreme case of the beginning of the game, $N=51$ (subtracting the known card from the pile top) and $n_j=5$, the number 
of card combinations are 2.3 million (in a deck with two jokers this would yield 2.9 million). Calculating the sum values of all hand combinations and creating a ratio for those $h_j>h_j$ is manageble with current computing power\footnote{On a MacBook Air 1.7 GHz Intel Core i7  for $N=45$, $n_j=5$ calculate a runtime of $\sim15$ seconds.}, but could be improved for scalability purposes.
\\ \\

\subsection{Threshold  as Prior}
If we are interested in calculating the success only after Yaniv has been declared, we can leverage this knowledge.  
E.g, examining the extreme case of Yaniv being declared at the first turn of the game, we can safely assume that player $i$ only has cards with values $1,2,3$. 
This does not change $N$ or $n_j$, but this motivates us to different approach using the chain rule with condition $U$:  
\begin{equation*}
P(h_j >h_i) = P(h_j >h_i |U)P(U) + P(h_j >h_i |\widetilde{U})P(\widetilde{U})
\end{equation*}, 
where $U:=  \forall \{C\}_{n_j} < t$, i.e, all possible unknown $N$ cards in set  $\{C\}_{n_j}$ that player $j$ could posses  are below a threshold $t$.   
The threshold  $t$ is the maximum card value that $j$ is allowed to posses in order to Assaf player $i$ a
and can be defined as a function of both the Yaniv value $y$ and $n_j$: $t=f(y, n_j)$. \\
Since $(\widetilde{U}$ means that at least one card value is above $t$ and $h_j$ is the Yaniv value we know that $P(h_j >h_i |\widetilde{U})=1$ reducing the calculation to 
\begin{equation*}
P(h_j >h_i) = P(h_j >h_i |U)P(U) + 1 - P(U)
\end{equation*}.   

This means that as long as that we can calculate $P(U)$ the only brute force required is to calculate $ P(h_j >h_i |U)$, 
which, by definition of $U$ means that we (practically in most cases substantially) reduced to a subset of $C$ from $N$.


\subsection{Determining Threshold $t_{n_j}(h_i)$}
The threshold is a key element in reducing the combinatorial search space, which may be substantial if $N$ is large.  
Hence our objective is to find a threshold $t$ such that $P(h_j >h_i |\widetilde{U})=1$, i.e, if player $j$ has a card $>t$ they 
will definitely lose to player $i$.  

E.g, if player $j$ possesses one card $n_j=1$,  in order to ensure $h_j > h_i$, if they have a card with value $h_i + 1$ they will lose. 
If $n_j=2$ the thresh is reduced to $h_i$. Generalising to $n_j$ cards we find:  
 
\begin{equation*}
t_{n_j}(h_i) = \text{max}(0, \ h_i - n_j + 2)
\end{equation*}  
If jokers are in play we this threshold if further relaxed to  $t_{n_j}(h_i) = \text{max}(0,\  h_i - n_j + 3)$. 
\\
We are now ready to calculate the probability that all cards are below $t_{n_j}$ $P(U)$. 

\subsection{Calculating $P(U)=P(\forall \{C\} <  t_{n_j})$}
Our objective is now to calculate the probability that all $n_j$ cards in player $j$'s possession are smaller than $t_{n_j}$, 
given $N$ unknown cards. To this extent we will convert all $N$ card values $v$ to binary values $b$: $b= v < t_{n_j}$. 

\[
   b= 
\begin{cases}
    1,& v < t_{n_j}\\
    0,              & \text{otherwise}
\end{cases}
\].
With this we can now use the hypergeometric distribution:

\begin{equation*}
P(U) = \frac{  \binom{K}{k}  \binom{N-K}{n_{j} - k} }{  \binom{N}{n_j} }
\end{equation*}, 
where $K=\sum_{m=1}^{C}b_m$ is the number of successes in the unknown cards $C$ and $k=\sum_{m=1}^{n_j}b_m$ is the number of successes in player $j$'s hand. 

Since we are interested in $k=n_j$ we obtain

\begin{equation*}
P(U) = \frac{  \binom{K}{n_j}  }{  \binom{N}{n_j} } = \frac{K! \left(N-n_j \right)! }{N!\left(K-n_j \right)!}
\end{equation*} 



\subsection{Prior Knowledge: Leveraging the Number of Cards}


\subsection{Test Subsection}
test
\subsubsection{Test Subsubsection}
test
\paragraph{Test Modified Paragraph}
test




We further simplify the task with the prior 
information of the number of cards possessed by player $j$ $n_{\mathrm{cards}_j}$, reducing the task to $P(H_j>H_i| n_{\mathrm{cards}_j})$. 
\\ \\ \\ \\ \\

Assuming $N_{\mathrm{unk}_i}$ cards that are unknown to player $i$ there are $\frac{N_{\mathrm{unk}_i}!}{4}$ combinations.




Probability that player $i$ has the lowest hand value $H_i$:  \\
(1) $P_{i}(H_i = \mathrm{min}\{H\}) = \Pi_{j!=i}^{N} P(H_j > H_i)$
\\ \\
Probability that player $j$ has a higher hand than player $i$  \\
(2) $P(H_j>H_i) = ?$
\\ \\
Rainman equation:\\
Probability that player $i$ has the lowest hand value $H_i$\\ given the number of cards in hand of each player $n_{\mathrm{cards}}$ and known cards of $j$ by $i$ $C_{j,i}$
\\
(3) $P_{i}(H_i = \mathrm{min}\{H\}) = \Pi_{j!=i}^{N} P(H_j > H_i|n_{\mathrm{cards}}, C_{j,i})$
\\
Mapping $T(n_{\mathrm{cards} })$ that is the threshold minimum possible given the number of cards in hand.\\ $P(H_j > H_i|n_{\mathrm{cards}})$
\\
(4) $P(H_j > H_i|n_{\mathrm{cards}}, C_{j,i})=\\ P(\mathrm{at least one card above T}) + P(\mathrm{all below })$ 
\end{document}